\documentclass[10 pt]{report}
\usepackage[total={5.9in, 8.7in}]{geometry}
\usepackage[T1]{fontenc}
\usepackage[utf8]{inputenc}
\usepackage[scaled=0.9]{helvet}
\renewcommand{\familydefault}{\sfdefault}
\usepackage[defaultmono]{droidsansmono}
\usepackage{CJKutf8}
\usepackage{pxrubrica}
\usepackage{kotex}
\usepackage{dhucs-trivcj}
\usepackage{multirow}
\usepackage{multicol}
\setlength{\columnsep}{0.5cm}
\usepackage{colortbl} % add colour to tables
\usepackage{tabularx} % superior tabular
\usepackage{imakeidx}
\usepackage{hyperref}
\hypersetup{%
    colorlinks = true,
    linkcolor = ocean,
    filecolor = aqua,
    urlcolor = aqua,
    pdfpagemode = FullScreen
}%

% Custom LGBT Colours
\usepackage{xcolor}  % Add some colour to life
% L
\definecolor{lavender}{HTML}{D783FF} % lavender
\definecolor{bubblegum}{HTML}{FF85FF} % bubblegum
\definecolor{carnation}{HTML}{FF8AD8} % carnation
\definecolor{salmon}{HTML}{FF7E79} % salmon
\definecolor{maraschino}{HTML}{FF2600} % maraschino
\definecolor{cayenne}{HTML}{941100} % cayenne
\definecolor{mocha}{HTML}{945200} % mocha
\definecolor{tangerine}{HTML}{FF9300} % tangerine
\definecolor{cantaloupe}{HTML}{FFD479} % cantaloupe
\definecolor{plum}{HTML}{942193} % plum
\definecolor{maroon}{HTML}{941751} % maroon

% G
\definecolor{asparagus}{HTML}{929000} % asparagus
\definecolor{fern}{HTML}{4F8F00} % fern
\definecolor{teal}{HTML}{009193} % teal
\definecolor{flora}{HTML}{73FA79} % flora
\definecolor{honeydew}{HTML}{D4FB79} % honeydew
\definecolor{seafoam}{HTML}{00FA92} % seafoam

% B
\definecolor{turquoise}{HTML}{00FDFF} % turquoise
\definecolor{ocean}{HTML}{005493} % ocean
\definecolor{midnight}{HTML}{011993} % midnight
\definecolor{eggplant}{HTML}{531B93} % eggplant

\definecolor{aqua}{HTML}{0096FF} % aqua
\definecolor{blueberry}{HTML}{0433FF} % blueberry
\definecolor{spindrift}{HTML}{73FCD6} % spindrift
\definecolor{orchid}{HTML}{7A81FF} % orchid

% Nord - https://www.nordtheme.com/docs/colors-and-palettes
\definecolor{night1}{HTML}{2E3440} % nord0
\definecolor{night2}{HTML}{3B4252} % nord1
\definecolor{night3}{HTML}{434C5E} % nord2
\definecolor{night4}{HTML}{4C556A} % nord3
% no snowstorm (nord4~nord6)
\definecolor{frost1}{HTML}{8FBCBB} % nord7
\definecolor{frost2}{HTML}{88C0D0} % nord8
\definecolor{frost3}{HTML}{81A1C1} % nord9
\definecolor{frost4}{HTML}{5E81AC} % nord10
\definecolor{aurorablue}{HTML}{5E81AC} % nord10
\definecolor{aurorared}{HTML}{BF616A} % nord11
\definecolor{auroraorange}{HTML}{D08770} % nord12
\definecolor{aurorayellow}{HTML}{EBCB8B} % nord13
\definecolor{auroragreen}{HTML}{A3BE8C} % nord14
\definecolor{aurorapurple}{HTML}{B48EAD} %n nord15

%---- Macros ----
\newcommand{\furi}[2][j]{\ruby[#1]{#2}}
\newcommand{\jpxm}[1]{\begin{CJK}{UTF8}{ipxm}#1\end{CJK}}
\newcommand{\jpxg}[1]{\begin{CJK}{UTF8}{ipxg}#1\end{CJK}}
\newcommand{\keyword}[1]{\textcolor{cayenne}{#1}}

% Index building
\makeindex[%
    columns=1,
    title=\Huge\jpxg{グロッサリー},
    % options= -s style.ist
]

% Entry
\newcommand{\jp}[3]{%
    \begin{CJK}{UTF8}{ipxm}
        \section*{\textcolor{night4}{#1}\index{#2} (#2)}%
    \end{CJK}

    \noindent\textcolor{ocean}{\rm\small{#3}}\vspace{2em}
    }
\newcommand{\eg}[2]{%
    \begin{CJK}{UTF8}{ipxm}%
        \noindent例) #1%
    \end{CJK}

    \noindent\hspace{1.6em}\textrm{#2}\vspace{1.3em}
    }
\newcommand{\egfinal}[2]{%
    \begin{CJK}{UTF8}{ipxm}%
        \noindent例) #1%
    \end{CJK}

    \noindent\hspace{1.6em}\textrm{#2}\vspace{1.35em}
    \hrule\vspace{1.35em}
    }

% \newcommand{\idx}[1]{\index{\jp{#1}}}

%---- Basic Title Info ----
\title{\Huge\jpxm{聞いていた単語\\そのコレクション}}
\author{\sfレオザライオン}
\date{}

%---- Main Document ----
\begin{document}
\maketitle
\pagebreak
\tableofcontents
\pagebreak

\chapter{\jpxm{その\furi{口}{くち}}}
\begin{multicols*}{2}
\jp{口が重い}{くちがおもい}{입이 무겁다, 과묵하다}

\egfinal{\furi{男}{おとこ}は\keyword{\furi{口}{くち}が\furi{重}{おも}く}なければならない}{사나이는 입이 무거워야 해.}

\jp{口が上手い}{くちがうまい}{말을 잘 하다, 말 솜씨가 좋다}

\egfinal{\keyword{\furi{口}{くち}が\furi{上|手}{う|ま}い}\furi{者}{しゃ}は\furi{嘘}{うそ}も\furi{上|手}{う|ま}い。}{말이 많은 자는 거짓말도 많다.}

\jp{口が煩い}{くちがうるさい}{말이 많은, (평판이) 시끌하다}

\eg{\furi{怠}{なま}け\furi{者}{もの}は、えてして\keyword{\furi{口}{くち}が\furi{煩}{うるさ}い}}{게으름뱅이는 대개 말이 많다}

\egfinal{\furi{世|間}{せ|けん}(の\keyword{\furi{口}{くち}})\keyword{が\furi{煩}{うるさ}い}}{세상 소문이 성가시다}

\jp{口が掛かる}{くちがかかる}{%
    1. (손님, 친구의) 부름을 받다\\
    2. (일의) 권유를 받다\\
    3. 입에 오르다, 구설에 오르다
}

\eg{\furi{麻|雀}{マー|ジャン}の\keyword{\furi{口}{くち}が\furi{掛}{か}かる}}{마작을 하자고 (친구가) 부르다}

\eg{\furi{昨日}{きのう}\furi{就|職}{しゅう|しょく}の\keyword{\furi{口}{くち}が\furi{掛}{か}かった。}}{어제 취직 권유를 받았어.}

\egfinal{\furi{彼}{かれ}の\keyword{\furi{口}{くち}に\furi{掛}{か}か}ってはどうにもならない。}{그의 입에 오르면 어쩔 도리가 없어.}

\jp{口が軽い}{くちがかるい}{(입, 주둥이가) 가볍다, 싸다, 헤프다}

\egfinal{\keyword{\furi{口}{くち}の\furi{軽}{かる}い}のが\furi{欠|点}{けっ|てん}[\furi{問|題}{もん|だい}]だ。}{입이 싸서 탈이다.}

\jp{口が過ぎる}{くちがすぎる}{(말이) 과하다, 지나치다, 건방지다}

\egfinal{\furi{年長者}{ねんちょうしゃ}に\furi{向}{む}かって\furi{少}{すこ}し\furi{口}{くち}が\furi{過}{す}ぎるぞ。}{어른한테 말이 좀 과한거 아냐?}

\jp{口当(た)り}{くちあたり}{%
    1. 입맛, 입에 닫는 느낌, 구미(口味)\\
    2. 응대(應對)
    }

\eg{\keyword{\furi{口}{くち}\furi{当}{あた}り}のいい\furi{酒}{さけ}}{입에 착 맞는 술 (=입에 당기는 술)}

\egfinal{\keyword{\furi{口}{くち}\furi{当}{あた}り}のいい[\furi{柔}{やわ}らかい]\furi{人}{ひと}}{응대를 잘 하는 사람}


\jp{口に会う}{くちにあう}{입(맛)에 맞는}

\egfinal{%
    \keyword{\furi{口}{くち}に\furi{合}{あ}う}\furi{和|食}{わ|しょく}が\furi{最|高}{さい|こう}。
    }{입에 맞는 일식(日食)이 최고.}

\jp{口にする}{くちにする}{%
    1. 입에 담는, 말하는\\
    2. 먹는, 입에 대는}

\eg{%
    ひょいひょいと\furi{冗|談}{じょう|だん}を\keyword{\furi{口}{くち}にする}
    }{이따금 농담을 하다}

\egfinal{%
    \furi{酒}{さけ}は\keyword{\furi{口}{くち}にしない}。
    }{술은 입에 대지 않아. (= 술 안 마셔.)}

\jp{口に出す}{くちにだす}{입밖에 내는, 말을 꺼내는}

\egfinal{\keyword{\furi{口}{くち}に\furi{出}{だ}す}べきことではない。
    }{입 밖에 낼 일이 아니야.}

\end{multicols*}
\pagebreak

\chapter{\jpxm{その手}}
\pagebreak

\chapter{\jpxm{その取り}}
\jp{草取り}{くさとり}{풀뽑기, 잡초 뽑기}

\eg{
    \furi{父}{ちち}は\furi{私}{わたし}に\furi{庭}{にわ}の\keyword{\furi{草}{くさ}\furi{取}{と}り}をやらせた。
    }{
    아빠는 나한테 풀뽑기를 하라고 했다.
    }
\pagebreak

\chapter{}
\jp{受け止める}{うけとめる}{%
    받아내는, 받아들이는, 막아내는
    }

\eg{\furi{忠|告}{ちゅう|こく}を\furi{謙|虚}{けん|きょ}に\furi{受}{う}け\furi{止}{と}める}{충고를 겸허히 받아들이다}

\pagebreak
\printindex
\end{document}

% \jp{}{}{%
%     }{%
%     }{%
%     }
